\newpage
\section{Заключение}

В результате выполнения курсовой работы были частично достигнуты её
сновные цели. По итогам получились следующие результаты:

\begin{itemize}

    \item Создан набор изменений ядра Linux для проекта SEAndroid
        реализующий расстановку контрольных точек для динамического
        изменения контекстов SELinux. Реализация заточена под
        процессорную архитектуру x86.

    \item Рассмотрены идеи и подходы для переноса реализации на
        архитектуру ARM.

    \item Работа реализованной системе была продемонстрирована на
        модельном примере приложения, содержащем уязвимость.

\end{itemize}

Из-за технической сложности не удалось построить рабочую реализацию на
ARM. После выпуска системы \cite{uprobes} на ARM целесообразно применить
описанные в данной работе подходы для построения реализации для
архитектуры, используемой в большинстве современных мобильных устройств.

Созданная реализация направлена на использование с системными сервисами
Android. В данной работе реализация была продемонстрирована на модельном
примере, но необходимо будет дополнительно показать применимость подхода
на реальной уязвимости. 

Существующая система также требует разметки контрольными точками
вручную. Поэтому предлагается провести интеграцию результатов данной
работы с результатами \cite{mso} для автоматизации процесса разметки.

\newpage

\begin{thebibliography}{99}
\bibitem{SEOF} 
Официальная документация SELinux [HTML] 

(\url{http://www.nsa.gov/research/selinux/docs.shtml})

\bibitem{AppArmor} 
Документация по проекту AppArmor [HTML]
(\url{http://en.opensuse.org/AppArmor\_Geeks})

\bibitem{LSM} 
Chris Wright, Crispin Cowan, James Morris, Stephen Smalley, Greg
Kroah-Hartman
Linux Security Modules: General Security Support for the Linux Kernel 
// USENIX, 2002

\bibitem{pax} 
Сайт проекта GRSecurity [HTML] 
(\url{http://pax.grsecurity.net/})

\bibitem{grsecurity}
Bradley Spengler Increasing Performance and Granularity
in Role-based Control System [PDF]
(\url{http://www.grsecurity.net/researchpaper.pdf}).

\bibitem{selinux2}
 Redhat SELinux Guide [HTML]
 (\url{http://www.redhat.com/docs/manuals/enterprise/RHEL-4-Manual/selinux-guide}).

\bibitem{mso}
Беззубцев С.О. Гамаюнов Д.Ю. Горнак Т.А. Сапожников А.В.
Сахаров Ф.В. Контроль безопасного поведения приложений
с помощью поведенческих моделей // Методы и средства обработки информации, 439-444, 2009

\bibitem{LDD}
Jonathan Corbet, Greg Kroah-Hartman, Allesandro Rubini Linux Device Drivers, O'Reilly, 2005. 640 c.

\bibitem{ULK} 
Daniel P. Bovet, Marco Cesati, Understanding the Linux Kernel, O'Reilly, 2002, 568 c.
 
\bibitem{utrace}
utrace
(\url{http://people.redhat.com/roland/utrace/})

\bibitem{uprobes}
uprobes
(\url{https://github.com/cuviper/linux-uprobes})

\bibitem{sacharov}
Сахаров Ф.В. Интеграция средств гранулярного контроля безопасности
поведения приложений в ОС Линукс. Москва, 2011

\bibitem{bush}
Бушмакин П.С. Контроль поведения потоков исполнения в много-поточных
приложениях NPTL с помощью SELinux. Москва, 2012


\bibitem{wedge}
Andrea Bittau, Petr Marchenko, Mark Handley, Brad Karp,
Splitting Applications into Reduced-Privilege Compartments [HTML]
(\url{http://www.usenix.org/event/nsdi08/tech/full\_papers/bittau/bittau\_html/}) 

\end{thebibliography}

\bigskip
\begin{comment}
\section {Заключение} 

В процессе решения поставленной задачи 
с использованием результатов работы прошлого 
года была разработана система контроля 
поведения приложения, различающая его 
внутренние состояния и корректирующая
относительно изменений внутренних состояний
ограничения, накладываемые системой SELinux
на поведение приложения. Данная система была 
реализована в качестве модуля ядра. При этом, 
изменения в самом ядре Linux являются минимальными, 
а утилиты пользовательского пространства, входящие
в проект SELinux остаются неизменными. Это позволяет 
легко поддерживать решение в процессе выхода новых 
версий ядра Linux. 

Кроме этого, реализованное решение достаточно 
слабо зависит от SELinux, используя лишь функции 
изменения контекста приложения в процессе исполнения.
Это позволит в случае необходимости легко 
перенести реализованную систему на другие системы 
контроля поведения приложений, существующие в Linux
такие, как AppArmor и Tomoyo.

Исполнимый код наблюдаемого приложения размечается 
контрольными точками, которые используются для 
разграничения пространства его внутренних состояний. 
Расставленные в коде контрольные точки не могут быть 
изменены или удалены злоумышленником, как не могут 
быть добавлены и новые контрольные точки. Это позволяет 
считать информацию о смене состояний надежной. 

На данный момент из-за высокой технической сложности 
не удалось привести требуемый пример уязвимого приложения 
или создать такой пример искуственно. Такой пример 
крайне желателен для обоснования актуальности проделанной
работы и дальнейшие усилия будут сконцентрированы именно 
на этом. 


\end{comment}

\newpage

\begin{thebibliography}{99}
\bibitem{SEOF} 
Официальная документация SELinux [HTML] 

(\url{http://www.nsa.gov/research/selinux/docs.shtml})

\bibitem{AppArmor} 
Документация по проекту AppArmor [HTML]

(\url{http://en.opensuse.org/AppArmor\_Geeks})

\bibitem{pax} 
Сайт проекта GRSecurity [HTML] 
(\url{http://pax.grsecurity.net/})

\bibitem{grsecurity}
Bradley Spengler Increasing Performance and Granularity
in Role-based Control System [PDF]
(\url{http://www.grsecurity.net/researchpaper.pdf}).

\bibitem{selinux2}
 Redhat SELinux Guide [HTML]
 (\url{http://www.redhat.com/docs/manuals/enterprise/RHEL-4-Manual/selinux-guide}).

\bibitem{mso}
Беззубцев С.О. Гамаюнов Д.Ю. Горнак Т.А. Сапожников А.В.
Сахаров Ф.В. Контроль безопасного поведения приложений
с помощью поведенческих моделей // Методы и средства обработки информации, 439-444, 2009

\bibitem{LDD}
Jonathan Corbet, Greg Kroah-Hartman, Allesandro Rubini Linux Device Drivers, O'Reilly, 2005. 640 c.

\bibitem{ULK} 
Daniel P. Bovet, Marco Cesati, Understanding the Linux Kernel, O'Reilly, 2002, 568 c.
 
\bibitem{utrace}
utrace
(\url{http://people.redhat.com/roland/utrace/})

\bibitem{sacharov}
Сахаров Ф.В. Интеграция средств гранулярного контроля безопасности
поведения приложений в ОС Линукс. Москва, 2011

\bibitem{bush}
Бушмакин П.С. Контроль поведения потоков исполнения в много-поточных
приложениях NPTL с помощью SELinux. Москва, 2012


\bibitem{wedge}
Andrea Bittau, Petr Marchenko, Mark Handley, Brad Karp,
Splitting Applications into Reduced-Privilege Compartments [HTML]
(\url{http://www.usenix.org/event/nsdi08/tech/full\_papers/bittau/bittau\_html/}) 

\bibitem{ndss}
Sven Bugiel , Lucas Davi, Alexandra Dmitrienko, Thomas Fischer, Ahmad-Reza
Sadeghi1, Bhargava Shastry
Towards Taming Privilege-Escalation Attacks on Android

\end{thebibliography}

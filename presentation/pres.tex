\documentclass{beamer}
\usepackage[english,russian]{babel}
\usepackage[utf8]{inputenc}
\usetheme{Singapore} %Goettingen,  Montpellier, Singapore

% \usetheme[pageofpages=of,% String used between the current page and the
%                          % total page count.
%           bullet=circle,% Use circles instead of squares for bullets.
%           titleline=true,% Show a line below the frame title.
%           alternativetitlepage=true,% Use the fancy title page.
%           titlepagelogo= msu, %logo-polito,% Logo for the first page.
%           %watermark=msu,%watermark-polito,% Watermark used in every page.
%           %watermarkheight=100px,% Height of the watermark.
%           %watermarkheightmult=2,% The watermark image is 4 times bigger
%                                 % than watermarkheight.
%           ]{Torino}
\usepackage{verbatim}

\usepackage{tikz}
\usetikzlibrary{shapes,arrows}

\usetikzlibrary{shapes.geometric,shapes.arrows,decorations.pathmorphing}
\usetikzlibrary{matrix,chains,scopes,positioning,arrows,fit,arrows,backgrounds}
\tikzstyle{block} = [rectangle, draw,  text width=20em, text centered, minimum height=4em, top color=white,
    bottom color=blue!50!black!20, draw=blue!40!black!60, very
    thick, rounded corners]
\tikzstyle{de} = [diamond, drawshade, top color=white,
    bottom color=blue!50!black!20, draw=blue!40!black!60, very
    thick, text centered, rounded corners, text width=6em, text badly centered, node distance=3cm, inner sep=0pt]

%     text width=5em, text centered, rounded corners, minimum height=4em]
 \tikzstyle{decision} = [diamond, drawshade,  top color=white,
    bottom color=blue!50!black!20, draw=blue!40!black!60, very
    thick, text width=4.5em, text badly centered, node distance=3cm, inner sep=0pt]

 

\tikzstyle{line} = [draw, -latex']
\tikzstyle{cloud} = [draw, ellipse,fill=red!20, node distance=3cm,
    minimum height=2em]
   
\tikzset{blue dotted/.style={draw=blue!50!white, line width=1pt,
                               dash pattern=on 1pt off 4pt on 6pt off 4pt,
                                inner sep=4mm, rectangle, rounded corners}}
\tikzstyle{blockwhite} = [rectangle, draw,  text width=5em, text centered, minimum height=4em, draw=white, fill=white]
\tikzstyle{bl} = [block, shade, top color=white,
    bottom color=blue!50!black!20, draw=blue!40!black!60, very
    thick, text width=20em, text centered, rounded corners, minimum height=4em]
\tikzstyle{blk} = [block, shade, top color=white,
    bottom color=blue!50!black!20, draw=blue!40!black!60, very
    thick, text width=9em, text centered, rounded corners, minimum height=4em]
\tikzstyle{blkred}=[blk, bottom color=red!50!black!20]
\tikzstyle{blkyell}=[blk, bottom color=yellow!50!black!20]
\tikzstyle{blb} = [block, shade, top color=white,
    bottom color=blue!50!black!20, draw=blue!40!black!60, very
    thick, text width=10em, text centered, rounded corners, minimum height=4em]

\def\watermarkoff{\def\beamer@decolines@watermark{}}
%\setbeamercolor{alert text}{fg=green}

\usepackage{wrapfig}
\begin{document}
\title{Интеграция средств гранулярного контроля безопасности поведения
        приложений в ОС Линукс}  
\author{\small{Выполнил: Фёдор Сахаров}\\ \small{Научный руководитель:  Гамаюнов Д. Ю.}}
\institute{Лаборатория вычислительных комплексов ВМК \\ МГУ имени М.В.Ломоносова}
\date{Москва, 2011} 
% Создание заглавной страницы
%\frame{\titlepage}
\begin{frame}[t, plain] 
\watermarkoff
 \titlepage
\end{frame}


\begin{comment}
\begin{frame}{Задача контроля поведения приложения на узлах.}
\watermarkoff
\begin{block}{Контроль поведения приложения на узлах}
 \begin{itemize}
  \item Использование уязвимости в программном обеспечении для получения доступа к системе.
  \item Повышение уровня безопасности за счет дополнения системы механизмами контроля поведения приложений. 
\end{itemize}
\end{block}

\begin{block}{Принципы на которых должен строится контроль поведения приложений}
    \begin{enumerate}
      \item Давать только необходимые привилегии и никакие другие.
      \item Давать привилегии только тем частям программы, которым они действительно нужны.
      \end{enumerate}
\end{block}
\end{frame}

\begin{frame}{Введение в предметную область. SELinux}
% картинка
\begin{center}
\scalebox{.40}{
\includegraphics{selinux.png}
}
\end{center}
\end{frame}


% \begin{frame}{Граф потока управления для ftpd}
% На картинке изображен упрощенный граф
% потока управления утилиты ftpd: два типа пользователей, различные действия различным пользователям.
% В качестве модели программы используется граф потока управления
% Политика SElinux задается для всей программы целиком. 
% На сладе приведены примеры тех действий которые могкт быть в политике
% Почему это плохо: 
%  -- уязвимость в считывании команды (показать на графе)
%  -- последствия эксплуатации
%  -- будет разрешено все, что задано политикой
% -- поэтому надо по-другому => переход к постановке задачи 

% \begin{columns}
%   \begin{column}{0.7\textwidth}
%     \begin{itemize}
% 	\item чтение конфигурационных файлов
% 	\item создение сокета, передача данных по сокету
% 	\item чтение, запись в файл data.
%     \end{itemize}
% \end{column}
% 
% \begin{column}{0.9\textwidth}
% \scalebox{.25}{
%  \input{miniftp_controlflow.tex}
% }
% \end{column}
% \end{columns}
% \end{frame}



\begin{frame}{Граф потока управления для ftpd}
% На картинке изображен упрощенный граф
% потока управления утилиты ftpd: два типа пользователей, различные действия различным пользователям.
% В качестве модели программы используется граф потока управления
% Политика SElinux задается для всей программы целиком. 
% На сладе приведены примеры правил политики (устно проговорить некоторые из них)
% Почему это плохо: 
%  -- уязвимость в считывании команды (показать на графе)
%  -- последствия эксплуатации
%  -- будет разрешено все, что задано политикой
% -- поэтому надо по-другому => переход к постановке задачи 

\begin{columns}
  \begin{column}{0.9\textwidth}
$\ \ \ \ can\_network(ftpd\_t)$\\
$\ \ \ \ can\_ypbind(ftpd\_t)$\\
$\ \ \ \ allow\ ftpd\_t\ self:unix\_stream\_socket;$\\
$\ \ \ \ allow\ ftpd\_t\ self:unix create\_socket\_perms;$\\
$\ \ \ \ allow\ ftpd\_t\ self:config\_file rw\_file\_perms;$\\
$\ \ \ \ allow\ ftpd\_t\ home\_user\_t:dir \{ getattr\ search \};$\\
$\ \ \ \ allow\ ftpd\_t\ data\_t:file \{ getattr\ read\ write\};$
\end{column}

\begin{column}{0.9\textwidth}
\scalebox{.25}{
 \input{miniftp_controlflow.tex}
}
\end{column}
\end{columns}
\end{frame}




\end{comment}

\begin{frame}{Постановка задачи}
 \begin{block}{Цель:}
  \small{Pасширение метода контроля поведения приложения, основанного на использовании политик, для увеличения гранулярности при контроле поведения приложений.}
 \end{block}
\begin{block}{Задача:}
\small{Разработка и реализация средства гранулярного контроля
        поведения разбитого на состояния приложения со стороны
        ядра Линукс.}
 \end{block}
 \begin{block}{Для достижения указанной цели необходимо:}
  \begin{itemize}
   \item \small{Составить обзор существующих систем безопасности уровня ядра ОС.}
   \item \small{Разработать набор инструментов для разметки
        приложений контрольными точками.}
   \item \small{Разработать подсистему ядра Линукс, способную
        переключать профиль SELinux приложения при изменении состояния.}
   \item \small{Провести испытание средства на уязвимом приложении.}
  \end{itemize}
 \end{block}
\end{frame}

\begin{frame}{Обзор систем безопасности уровня ядра ОС.}
В данной работе был сделан обзор 4 систем безопасности уровня ядра ОС.
Критерии обзора:

\begin{itemize}
 \item Реализованные модели безопасности.
 \item Возможность изменять матрицу доступа в процессе исполнения.
 \item Динамическая смена контекстов.
 \item Классы вредоносных действий, предотвращаемых системой.
\end{itemize}
 Обзор показал, что ни одна из существующих систем безопасности 
 уровня ядра ОС не предоставляет возможности динамически изменять
 контекст приложения.
 
 \end{frame}


\begin{frame}{Анализ поведения приложения с использованием контрольных точек}
Для решения поставленной задачи была реализована система,
позвляющая изменять контекст безопасности приложения в зависимости
от его внутреннего состояния.
Реализованная система отвечает следующим требованиям:
\begin{itemize}
\item Наблюдение и изменение контекстов производится
        <<прозрачно>> для приложения.
\item Наблюдаемое приложение не модифицируется.
\end{itemize}

\end{frame}

\begin{frame}{Изменения ядра ОС Линукс}
Основная версия ядра ОС Линукс не позволяет динамически
изменять контексты безопасности приложения.

Для решения поставленной задачи были внесены следующие
изменения в ядро Линукс:

\begin{itemize}
\item Возможность динамически изменять контексты приложений
        из модулей ядра
\item Возможность отслеживать события запуска новых процессов
        в системе
\end{itemize}
\end{frame}
 


 

% ----------------------------------------------------
\begin{frame}[fragile]{Контрольные точки}
Контрольной точкой считается адрес в сегменте кода
виртуального адресного пространства приложения.

Для наблюдения за попаданием исполнения на контрольные
точки были использованы подсистемы utrace и uprobes
ядра Linux.

Данные системы позволяют:
\begin{itemize}
\item Отслеживать события попадания исполнения на контрольные точки
\item Отслеживать системные вызовы и другие события в наблюдаемом приложении
\end{itemize}

\end{frame}

\begin{frame}{Результаты}
\begin{block}{Результаты}
\begin{itemize}
\item Проведен обзор существующих систем безопасности уровня ядра ОС.
\item Были внесены изменения в ядро Линукс, позволяющие производить
        динамическую смену контекстов в приложении.
\item Реализован модуль ядра, позволяющий контролировать поведение
        приложений с использованием контрольных точек и изменять
        контекст безопасности наблюдаемых приложений в зависимости от
        их внутреннего состояния.

\item Проведено испытание реализованного модуля на модельном примере
        уязвимого приложения.
\end{itemize}
\end{block}
\end{frame}

\begin{comment}
\begin{frame}{Характеристика реализации}
 
\begin{tabular}{| l | c | } 
\hline
  Язык программирования & C, Python\\ \hline
  Операционная система &  Linux \\ \hline  
  Кол-во строк кода & n \\ 
\hline
\end{tabular}
\end{frame}

\begin{frame}{Результаты}

\begin{block}{Результаты}
\begin{itemize}
 \item Проведен обзор и сравнительный анализ средств построения моделей нормального поведения приложений.
 \item Разработан метод автоматизированного построения нормального поведения приложений для ОС Linux и языка Си.
 \item Реализовано средство для приложений, написанных для ОС Linux на языке Си, позволяющие разбивать программу на набор блоков.
 \item Проведено испытание реализованного средства на уязвимом приложений.
\end{itemize}
\end{block}
\end{frame}
\end{comment}



% ----------------------------------------------------
  \begin{frame}{Спасибо за внимание. Вопросы?}
\begin{center}
\LARGE{Спасибо за внимание. Вопросы?}
\end{center}

   
  \end{frame}
% ----------------------------------------------------
 

\end{document}



\documentclass{beamer}
\usepackage[english,russian]{babel}
\usepackage[utf8]{inputenc}
%  \usetheme{Singapore} %Goettingen,  Montpellier, Singapore

\usetheme[pageofpages=of,% String used between the current page and the
                         % total page count.
          bullet=circle,% Use circles instead of squares for bullets.
          titleline=true,% Show a line below the frame title.
          alternativetitlepage=true,% Use the fancy title page.
          titlepagelogo= msu, %logo-polito,% Logo for the first page.
          watermark=msu,%watermark-polito,% Watermark used in every page.
          watermarkheight=100px,% Height of the watermark.
          watermarkheightmult=4,% The watermark image is 4 times bigger
                                % than watermarkheight.
          ]{Torino}

\usecolortheme{freewilly}

% \usetheme[pageofpages=of,% String used between the current page and the
%                          % total page count.
%           bullet=circle,% Use circles instead of squares for bullets.
%           titleline=true,% Show a line below the frame title.
%           alternativetitlepage=true,% Use the fancy title page.
%           titlepagelogo= msu, %logo-polito,% Logo for the first page.
%           %watermark=msu,%watermark-polito,% Watermark used in every page.
%           %watermarkheight=100px,% Height of the watermark.
%           %watermarkheightmult=2,% The watermark image is 4 times bigger
%                                 % than watermarkheight.
%           ]{Torino}
\usepackage{verbatim}
\usepackage{hyperref}
\hypersetup{
	unicode=true,
	pdftitle={
	},
	pdfauthor={},
	pdfkeywords={
	},
	colorlinks,
	citecolor=black,
	filecolor=black,
	linkcolor=black,
	urlcolor=blue
}

\usepackage{listings}
\lstloadlanguages{[LaTex]Tex, C++, Ada, Java, Modula-2, Delphi, Oberon-2}
 \lstset{ backgroundcolor=\color{white},
	  extendedchars=true, 
	  frame=tb
}
\lstloadlanguages{% check dokumentation for further languages ...
    %[visual]basic
        %pascal
        c
        %c++
        %xml
        %HTML
        %Java
}

\usepackage{tikz}
\usetikzlibrary{shapes,arrows}

\usetikzlibrary{shapes.geometric,shapes.arrows,decorations.pathmorphing}
\usetikzlibrary{matrix,chains,scopes,positioning,arrows,fit,arrows,backgrounds}
\tikzstyle{block} = [rectangle, draw,  text width=20em, text centered, minimum height=4em, top color=white,
    bottom color=blue!50!black!20, draw=blue!40!black!60, very
    thick, rounded corners]
\tikzstyle{de} = [diamond, drawshade, top color=white,
    bottom color=blue!50!black!20, draw=blue!40!black!60, very
    thick, text centered, rounded corners, text width=6em, text badly centered, node distance=3cm, inner sep=0pt]

%     text width=5em, text centered, rounded corners, minimum height=4em]
 \tikzstyle{decision} = [diamond, drawshade,  top color=white,
    bottom color=blue!50!black!20, draw=blue!40!black!60, very
    thick, text width=4.5em, text badly centered, node distance=3cm, inner sep=0pt]

 

\tikzstyle{line} = [draw, -latex']
\tikzstyle{cloud} = [draw, ellipse,fill=red!20, node distance=3cm,
    minimum height=2em]
   
\tikzset{blue dotted/.style={draw=blue!50!white, line width=1pt,
                               dash pattern=on 1pt off 4pt on 6pt off 4pt,
                                inner sep=4mm, rectangle, rounded corners}}
\tikzstyle{blockwhite} = [rectangle, draw,  text width=5em, text centered, minimum height=4em, draw=white, fill=white]
\tikzstyle{bl} = [block, shade, top color=white,
    bottom color=blue!50!black!20, draw=blue!40!black!60, very
    thick, text width=20em, text centered, rounded corners, minimum height=4em]
\tikzstyle{blk} = [block, shade, top color=white,
    bottom color=blue!50!black!20, draw=blue!40!black!60, very
    thick, text width=9em, text centered, rounded corners, minimum height=4em]
\tikzstyle{blkred}=[blk, bottom color=red!50!black!20]
\tikzstyle{blkyell}=[blk, bottom color=yellow!50!black!20]
\tikzstyle{blb} = [block, shade, top color=white,
    bottom color=blue!50!black!20, draw=blue!40!black!60, very
    thick, text width=10em, text centered, rounded corners, minimum height=4em]

\def\watermarkoff{\def\beamer@decolines@watermark{}}
%\setbeamercolor{alert text}{fg=green}

\usepackage{wrapfig}
\begin{document}
\title{Интеграция средств гранулярного контроля безопасности поведения
        приложений в ОС Линукс}  
\author{\small{Выполнил: Фёдор Сахаров}\\ \small{Научный руководитель: м.н.с., к.ф.-м.н. Гамаюнов Д. Ю.}}
\institute{Лаборатория вычислительных комплексов ВМК \\ МГУ имени М.В.Ломоносова}
\date{Москва, 2011} 
% Создание заглавной страницы
%\frame{\titlepage}
\begin{frame}[t, plain] 
\watermarkoff
 \titlepage
\end{frame}


\begin{comment}
\begin{frame}{Задача контроля поведения приложения на узлах.}
\watermarkoff
\begin{block}{Контроль поведения приложения на узлах}
 \begin{itemize}
  \item Использование уязвимости в программном обеспечении для получения доступа к системе.
  \item Повышение уровня безопасности за счет дополнения системы механизмами контроля поведения приложений. 
\end{itemize}
\end{block}

\begin{block}{Принципы на которых должен строится контроль поведения приложений}
    \begin{enumerate}
      \item Давать только необходимые привилегии и никакие другие.
      \item Давать привилегии только тем частям программы, которым они действительно нужны.
      \end{enumerate}
\end{block}
\end{frame}

\begin{frame}{Введение в предметную область. SELinux}
% картинка
\begin{center}
\scalebox{.40}{
\includegraphics{selinux.png}
}
\end{center}
\end{frame}


% \begin{frame}{Граф потока управления для ftpd}
% На картинке изображен упрощенный граф
% потока управления утилиты ftpd: два типа пользователей, различные действия различным пользователям.
% В качестве модели программы используется граф потока управления
% Политика SElinux задается для всей программы целиком. 
% На сладе приведены примеры тех действий которые могкт быть в политике
% Почему это плохо: 
%  -- уязвимость в считывании команды (показать на графе)
%  -- последствия эксплуатации
%  -- будет разрешено все, что задано политикой
% -- поэтому надо по-другому => переход к постановке задачи 

% \begin{columns}
%   \begin{column}{0.7\textwidth}
%     \begin{itemize}
% 	\item чтение конфигурационных файлов
% 	\item создение сокета, передача данных по сокету
% 	\item чтение, запись в файл data.
%     \end{itemize}
% \end{column}
% 
% \begin{column}{0.9\textwidth}
% \scalebox{.25}{
%  \begin{tikzpicture}[
    grow=right, 
    level 1/.style={sibling distance=3.5cm,level distance=5.2cm},
    level 2/.style={sibling distance=3.5cm, level distance=6.7cm},
    edge from parent/.style={very thick,draw=blue!40!black!60,
        shorten >=5pt, shorten <=5pt},
    edge from parent path={(\tikzparentnode.east) -- (\tikzchildnode.west)},
    kant/.style={text width=2cm, text centered, sloped},
    every node/.style={text ragged, inner sep=2mm},
    cir/.style={circle, shade, top color=white,
    bottom color=blue!50!black!20, draw=blue!40!black!60, very
    thick },
    cirr/.style={circle, shade, top color=white,
    bottom color=red!50!black!20, draw=red!40!black!60, very
    thick },
    dr/.style={diamond, draw, shade,  top color=white,
    bottom color=red!50!black!20, draw=red!40!black!60, very
    thick},
    d/.style={diamond, draw, shade,  top color=white,
    bottom color=blue!50!black!20, draw=blue!40!black!60, very
    thick}]
\node[cir] (1) {};

\node[dr, below of=1] (if1) {};
\node[cir, right of=if1] (exit1) {};
\node[cir, below of=if1] (203) {};
\node[dr, below of=203] (while1) {};
\node[d, below of=while1] (if9) {};
\node[d, below of=if9] (if2) {};
\node[cir, below of=if2] (5) {};

\node[d, below of=5] (if3) {};
\node[cir, right of=if3] (607) {};


\node[d, below of=if3] (if4) {};
\node[cir, right of=if4] (8) {};

\node[cir, below of=if4] (9) {};
\node[cirr, below of=9] (55) {};

\node[cir, below of=55, node distance=1.5cm] (10) {};

\node[dr, below of=10] (if5) {};
\node[cir, right of=if5] (12013) {};


\node[cir, below of=if5] (14) {};

\node[dr, below of=14] (if6) {};
\node[cir, right of=if6] (15016) {};

\node[dr, below of=if6] (if7) {};
\node[cir, right of=if7] (17018) {};

\node[dr, below of=if7] (if8) {};
\node[cir, right of=if8] (19020) {};

\node[cir, below of=if8] (21) {};

\node[cir, below of=21, node distance=1.5cm] (22) {};
\node[cirr, below of=22] (23) {};

\node[cirr, below of=23] (auth) {};

\node[d, below of=auth] (if10) {};
\node[cir, right of=if10] (exit24) {};

\node[d, below of=if10] (if11) {};
\node[cir, left of=if11] (return25) {};

\node[d, below of=if11] (if12) {};
\node[cir, right of=if12] (return26) {};
\node[cir, right of=if12, below of=if12] (exit27) {};


\node[d, below of=if12] (if13) {};
\node[d, below of=if13, left of=if13, node distance=3.6cm] (while2) {};
\node[d, below of=if13, right of=if13, node distance=3.6cm] (while3) {};

\node[d, below of=while3] (switch1) {};
\node[cirr, below of=switch1] (1help) {};
\node[cirr, left of=1help] (1add) {};
\node[cirr, left of=1add] (1list) {};
\node[cirr, right of=1help] (1show) {};
\node[cirr, right of=1show] (1desc) {};
\node[cir, right of=1desc] (1quit) {};

\node[cir, below of=1help] (1vir) {};

\node[d, below of=while2] (if14) {};

\node[d, below of=if14] (switch2) {};
\node[cirr, below of=switch2] (2help) {};
\node[cir, left of=2help] (2add) {};
\node[cirr, left of=2add] (2list) {};
\node[cirr, right of=2help] (2show) {};
\node[cir, right of=2show] (2desc) {};
\node[cir, right of=2desc] (2quit) {};

\node[cir, below of=2help] (2vir) {};

\draw[->] (1) -- (if1);
\draw[->] (if1) -- (exit1);
\draw[->] (if1) -- (203);
\draw[->] (203) -- (while1);
\draw[->] (while1) -- (if9);

\draw[->] (if9) -- (if2);
\draw[->] (if2) -- (5);
\draw[->] (5) -- (if3);
\draw[->] (if3) -- (if4);
\draw[->] (if4) -- (9);
\draw[->] (9) -- (55);
\draw[->] (55) -- (10);
\draw[->] (10) -- (if5);
\draw[->] (if5) -- (14);
\draw[->] (14) -- (if6);
\draw[->] (if6) -- (if7);

\draw[->] (if7) -- (if8);
\draw[->] (if8) -- (21);



\path (if9) edge[pil,->, bend right=45] node[auto] {} (while1);
\path (9) edge[pil,->, bend left=45] node[auto] {} (while1);

\draw[->] (if3) -- (607);
\draw[->] (607) -- (if4);

\draw[->] (if4) -- (8);
\draw[->] (8) -- (9);

\draw[->] (if5) -- (12013);

\draw[->] (if6) -- (15016);

\draw[->] (if7) -- (17018);

\draw[->] (if8) -- (19020);


\draw[->] (22) -- (23);

\draw[->] (23) -- (auth);
\draw[->] (auth) -- (if10);
\draw[->] (if10) -- (exit24);

\draw[->] (if10) -- (if11);
\draw[->] (if11) -- (return25);

\draw[->] (if11) -- (if12);
\draw[->] (if12) -- (return26);

\draw[->] (if12.south) -- (exit27);

\draw[->] (return26) -- (if13);
\draw[->] (return25) -- (if13);

\draw[->] (if13) -- (while2);
\draw[->] (if13) -- (while3);


\draw[->] (while3) -- (switch1);
\draw[->] (switch1) -- (1help);
\draw[->] (switch1) -- (1add);
\draw[->] (switch1) -- (1list);
\draw[->] (switch1) -- (1show);
\draw[->] (switch1) -- (1desc);
\draw[->] (switch1) -- (1quit);

\draw[->] (1help) -- (1vir);
\draw[->] (1add) -- (1vir);
\draw[->] (1list) -- (1vir);
\draw[->] (1show) -- (1vir);
\draw[->] (1desc) -- (1vir);

\path (1vir) edge[pil,->, bend left=100] node[auto] {} (while3);
\path (if14.west) edge[pil,->, bend left=45] node[auto] {} (while2);


\draw[->] (while2) -- (if14);
\draw[->] (if14) -- (switch2);

\draw[->] (switch2) -- (2help);
\draw[->] (switch2) -- (2add);
\draw[->] (switch2) -- (2list);
\draw[->] (switch2) -- (2show);
\draw[->] (switch2) -- (2desc);
\draw[->] (switch2) -- (2quit);

\draw[->] (2help) -- (2vir);
\draw[->] (2add) -- (2vir);
\draw[->] (2list) -- (2vir);
\draw[->] (2show) -- (2vir);
\draw[->] (2desc) -- (2vir);

\path (2vir) edge[pil,->, bend left=100] node[auto] {} (while2);

\draw[->] (21) -- (22) ;

% \node (parsercnf) [blue dotted, fit = (exit1) (if1) (if2) (if3) (if4) (if9) (while1) (1) (203) (5) (607) (8) (9)] {};
% 
% \node (socketconn) [blue dotted, fit = (if5) (if6) (if7) (if8)  (12013) (15016) (17018) (19020) (21) (10) (14)] {};
% 
% \node (getauthdata) [blue dotted, fit = (22) (23)] {};
% 
% \node (auth) [blue dotted, fit = (if10) (if12) (if11)  (24) (25) (26) (27) ] {};
% 
% 
% \node at (parsercnf.west) [above, inner sep=1cm] {funct\_parsercnf};
% \node at (socketconn.west) [above, inner sep=1cm] {funct\_socketconnect};
% \node at (getauthdata.west) [above, inner sep=1cm] {funct\_getauthdata};
% \node at (auth.west) [above, inner sep=1cm] {funct\_auth};
\end{tikzpicture}

% }
% \end{column}
% \end{columns}
% \end{frame}



\begin{frame}{Граф потока управления для ftpd}
% На картинке изображен упрощенный граф
% потока управления утилиты ftpd: два типа пользователей, различные действия различным пользователям.
% В качестве модели программы используется граф потока управления
% Политика SElinux задается для всей программы целиком. 
% На сладе приведены примеры правил политики (устно проговорить некоторые из них)
% Почему это плохо: 
%  -- уязвимость в считывании команды (показать на графе)
%  -- последствия эксплуатации
%  -- будет разрешено все, что задано политикой
% -- поэтому надо по-другому => переход к постановке задачи 

\begin{columns}
  \begin{column}{0.9\textwidth}
$\ \ \ \ can\_network(ftpd\_t)$\\
$\ \ \ \ can\_ypbind(ftpd\_t)$\\
$\ \ \ \ allow\ ftpd\_t\ self:unix\_stream\_socket;$\\
$\ \ \ \ allow\ ftpd\_t\ self:unix create\_socket\_perms;$\\
$\ \ \ \ allow\ ftpd\_t\ self:config\_file rw\_file\_perms;$\\
$\ \ \ \ allow\ ftpd\_t\ home\_user\_t:dir \{ getattr\ search \};$\\
$\ \ \ \ allow\ ftpd\_t\ data\_t:file \{ getattr\ read\ write\};$
\end{column}

\begin{column}{0.9\textwidth}
\scalebox{.25}{
 \begin{tikzpicture}[
    grow=right, 
    level 1/.style={sibling distance=3.5cm,level distance=5.2cm},
    level 2/.style={sibling distance=3.5cm, level distance=6.7cm},
    edge from parent/.style={very thick,draw=blue!40!black!60,
        shorten >=5pt, shorten <=5pt},
    edge from parent path={(\tikzparentnode.east) -- (\tikzchildnode.west)},
    kant/.style={text width=2cm, text centered, sloped},
    every node/.style={text ragged, inner sep=2mm},
    cir/.style={circle, shade, top color=white,
    bottom color=blue!50!black!20, draw=blue!40!black!60, very
    thick },
    cirr/.style={circle, shade, top color=white,
    bottom color=red!50!black!20, draw=red!40!black!60, very
    thick },
    dr/.style={diamond, draw, shade,  top color=white,
    bottom color=red!50!black!20, draw=red!40!black!60, very
    thick},
    d/.style={diamond, draw, shade,  top color=white,
    bottom color=blue!50!black!20, draw=blue!40!black!60, very
    thick}]
\node[cir] (1) {};

\node[dr, below of=1] (if1) {};
\node[cir, right of=if1] (exit1) {};
\node[cir, below of=if1] (203) {};
\node[dr, below of=203] (while1) {};
\node[d, below of=while1] (if9) {};
\node[d, below of=if9] (if2) {};
\node[cir, below of=if2] (5) {};

\node[d, below of=5] (if3) {};
\node[cir, right of=if3] (607) {};


\node[d, below of=if3] (if4) {};
\node[cir, right of=if4] (8) {};

\node[cir, below of=if4] (9) {};
\node[cirr, below of=9] (55) {};

\node[cir, below of=55, node distance=1.5cm] (10) {};

\node[dr, below of=10] (if5) {};
\node[cir, right of=if5] (12013) {};


\node[cir, below of=if5] (14) {};

\node[dr, below of=14] (if6) {};
\node[cir, right of=if6] (15016) {};

\node[dr, below of=if6] (if7) {};
\node[cir, right of=if7] (17018) {};

\node[dr, below of=if7] (if8) {};
\node[cir, right of=if8] (19020) {};

\node[cir, below of=if8] (21) {};

\node[cir, below of=21, node distance=1.5cm] (22) {};
\node[cirr, below of=22] (23) {};

\node[cirr, below of=23] (auth) {};

\node[d, below of=auth] (if10) {};
\node[cir, right of=if10] (exit24) {};

\node[d, below of=if10] (if11) {};
\node[cir, left of=if11] (return25) {};

\node[d, below of=if11] (if12) {};
\node[cir, right of=if12] (return26) {};
\node[cir, right of=if12, below of=if12] (exit27) {};


\node[d, below of=if12] (if13) {};
\node[d, below of=if13, left of=if13, node distance=3.6cm] (while2) {};
\node[d, below of=if13, right of=if13, node distance=3.6cm] (while3) {};

\node[d, below of=while3] (switch1) {};
\node[cirr, below of=switch1] (1help) {};
\node[cirr, left of=1help] (1add) {};
\node[cirr, left of=1add] (1list) {};
\node[cirr, right of=1help] (1show) {};
\node[cirr, right of=1show] (1desc) {};
\node[cir, right of=1desc] (1quit) {};

\node[cir, below of=1help] (1vir) {};

\node[d, below of=while2] (if14) {};

\node[d, below of=if14] (switch2) {};
\node[cirr, below of=switch2] (2help) {};
\node[cir, left of=2help] (2add) {};
\node[cirr, left of=2add] (2list) {};
\node[cirr, right of=2help] (2show) {};
\node[cir, right of=2show] (2desc) {};
\node[cir, right of=2desc] (2quit) {};

\node[cir, below of=2help] (2vir) {};

\draw[->] (1) -- (if1);
\draw[->] (if1) -- (exit1);
\draw[->] (if1) -- (203);
\draw[->] (203) -- (while1);
\draw[->] (while1) -- (if9);

\draw[->] (if9) -- (if2);
\draw[->] (if2) -- (5);
\draw[->] (5) -- (if3);
\draw[->] (if3) -- (if4);
\draw[->] (if4) -- (9);
\draw[->] (9) -- (55);
\draw[->] (55) -- (10);
\draw[->] (10) -- (if5);
\draw[->] (if5) -- (14);
\draw[->] (14) -- (if6);
\draw[->] (if6) -- (if7);

\draw[->] (if7) -- (if8);
\draw[->] (if8) -- (21);



\path (if9) edge[pil,->, bend right=45] node[auto] {} (while1);
\path (9) edge[pil,->, bend left=45] node[auto] {} (while1);

\draw[->] (if3) -- (607);
\draw[->] (607) -- (if4);

\draw[->] (if4) -- (8);
\draw[->] (8) -- (9);

\draw[->] (if5) -- (12013);

\draw[->] (if6) -- (15016);

\draw[->] (if7) -- (17018);

\draw[->] (if8) -- (19020);


\draw[->] (22) -- (23);

\draw[->] (23) -- (auth);
\draw[->] (auth) -- (if10);
\draw[->] (if10) -- (exit24);

\draw[->] (if10) -- (if11);
\draw[->] (if11) -- (return25);

\draw[->] (if11) -- (if12);
\draw[->] (if12) -- (return26);

\draw[->] (if12.south) -- (exit27);

\draw[->] (return26) -- (if13);
\draw[->] (return25) -- (if13);

\draw[->] (if13) -- (while2);
\draw[->] (if13) -- (while3);


\draw[->] (while3) -- (switch1);
\draw[->] (switch1) -- (1help);
\draw[->] (switch1) -- (1add);
\draw[->] (switch1) -- (1list);
\draw[->] (switch1) -- (1show);
\draw[->] (switch1) -- (1desc);
\draw[->] (switch1) -- (1quit);

\draw[->] (1help) -- (1vir);
\draw[->] (1add) -- (1vir);
\draw[->] (1list) -- (1vir);
\draw[->] (1show) -- (1vir);
\draw[->] (1desc) -- (1vir);

\path (1vir) edge[pil,->, bend left=100] node[auto] {} (while3);
\path (if14.west) edge[pil,->, bend left=45] node[auto] {} (while2);


\draw[->] (while2) -- (if14);
\draw[->] (if14) -- (switch2);

\draw[->] (switch2) -- (2help);
\draw[->] (switch2) -- (2add);
\draw[->] (switch2) -- (2list);
\draw[->] (switch2) -- (2show);
\draw[->] (switch2) -- (2desc);
\draw[->] (switch2) -- (2quit);

\draw[->] (2help) -- (2vir);
\draw[->] (2add) -- (2vir);
\draw[->] (2list) -- (2vir);
\draw[->] (2show) -- (2vir);
\draw[->] (2desc) -- (2vir);

\path (2vir) edge[pil,->, bend left=100] node[auto] {} (while2);

\draw[->] (21) -- (22) ;

% \node (parsercnf) [blue dotted, fit = (exit1) (if1) (if2) (if3) (if4) (if9) (while1) (1) (203) (5) (607) (8) (9)] {};
% 
% \node (socketconn) [blue dotted, fit = (if5) (if6) (if7) (if8)  (12013) (15016) (17018) (19020) (21) (10) (14)] {};
% 
% \node (getauthdata) [blue dotted, fit = (22) (23)] {};
% 
% \node (auth) [blue dotted, fit = (if10) (if12) (if11)  (24) (25) (26) (27) ] {};
% 
% 
% \node at (parsercnf.west) [above, inner sep=1cm] {funct\_parsercnf};
% \node at (socketconn.west) [above, inner sep=1cm] {funct\_socketconnect};
% \node at (getauthdata.west) [above, inner sep=1cm] {funct\_getauthdata};
% \node at (auth.west) [above, inner sep=1cm] {funct\_auth};
\end{tikzpicture}

}
\end{column}
\end{columns}
\end{frame}




\end{comment}

\begin{frame}{Стандартная модель безопасности UNIX и её расширения}

\begin{block}{Стандартная модель безопасности в UNIX:}
\begin{itemize}
\item Единицей контроля является пользователь.
\item Каждый процесс, запущенный пользователя обладает его правами.
\end{itemize}
\end{block}

\begin{block}{Расширения(SELinux, AppArmor):}
\begin{itemize}
\item Единицей контроля является процесс.
\item Более высокий уровень детализации описания привилегий.
\item Каждому процессу даются только те привилегии, которые необходимы для нормального исполнения.
\end{itemize}
\end{block}
\end{frame}

\begin{frame}{Постановка задачи}
  \begin{small}
\textcolor{blue}{\textbf{Цель:}} расширение метода контроля поведения
        приложения, основанного на использовании политик,
        для увеличения гранулярности при контроле поведения приложений.

\textcolor{blue}{\textbf{Задача:}} разработка и реализация средства гранулярного контроля
        поведения разбитого на состояния приложения со стороны
        ядра Линукс.

\end{small}
 \begin{block}{Для достижения указанной цели необходимо:}
  \begin{small}
  \begin{itemize}
   \item {Составить обзор существующих систем безопасности уровня ядра ОС.}
   \item {Разработать набор инструментов для разметки
        приложений контрольными точками.}
   \item {Разработать подсистему ядра Линукс, способную
        переключать профиль SELinux приложения при изменении состояния.}
   \item {Провести испытание средства на уязвимом приложении.}
  \end{itemize}
  \end{small}
 \end{block}
\end{frame}

\begin{frame}{Обзор систем безопасности уровня ядра ОС.}
В данной работе был сделан обзор четырех систем безопасности уровня ядра ОС.

\begin{block}{Критерии обзора:}

\begin{itemize}
 \item Реализованные модели безопасности.
 \item Возможность изменять матрицу доступа в процессе исполнения.
 \item Возможность динамически изменять контексты приложений.
 \item Классы вредоносных действий, предотвращаемых системой.
\end{itemize}
\end{block}
\end{frame}

\begin{frame}[fragile]{Результаты обзора}

Обзор показал, что ни одна из существующих систем безопасности
уровня ядра ОС не предоставляет возможности динамически изменять
контекст приложения.

\begin{tiny}
\begin{center}
\begin{tabular}{|p{1.5cm}|p{1.0cm}|p{2cm}|p{2cm}|p{3cm}|} 
\hline
Система Безопасности & Модели & Возможность менять матрицу 
доступа в процессе выполнения & 
Динамическая смена контекстов & Классы вредоносных действий,
предотвращаемых системой \\
\hline 
SELinux & TE, MAC, RBAC & Существует & Существует & 
Минимизация ущерба от успешных атак \\
\hline
AppArmor & TE & Существует & Существует только для сервера Apache & 
Минимизация ущерба от успешных атак \\ 
\hline 
GRSecurity & ACL & Отсутствует & Отсутствует & Внедрение кода в приложение 
и его исполнение, изменение нормального течения исполнения приложения, 
исполнение с повышенными привилегиями \\ 
\hline 
Trusted BSD & MAC, ACL, RBAC, Audit & Отсутствует & Существует & 
Минимизация ущерба от успешных атак \\ 
\hline
\end{tabular} 
\end{center} 
\end{tiny} 
 
\end{frame}


\begin{frame}{Анализ поведения приложения с использованием контрольных точек}
Для решения поставленной задачи была реализована система,
позволяющая изменять контекст безопасности приложения в зависимости
от его внутреннего состояния.
Реализованная система отвечает следующим требованиям:
\begin{itemize}
\item Наблюдение и изменение контекстов производится
        <<прозрачно>> для приложения.
\item Наблюдаемое приложение не модифицируется.
\end{itemize}

\end{frame}
\begin{frame}{Изменения ядра ОС Линукс}
Основная версия ядра ОС Линукс не позволяет динамически
изменять контексты безопасности приложения.

Для решения поставленной задачи были внесены следующие
изменения в ядро Линукс:

\begin{itemize}
\item Возможность динамически изменять контексты приложений
        из модулей ядра
\item Возможность отслеживать события запуска новых процессов
        в системе
\end{itemize}
\end{frame}
 


 

% ----------------------------------------------------
\begin{frame}[fragile]{Контрольные точки}
Контрольной точкой считается адрес в сегменте кода
виртуального адресного пространства приложения.

Для наблюдения за попаданием исполнения на контрольные
точки были использованы подсистемы utrace и uprobes
ядра Linux.

Данные системы позволяют:
\begin{itemize}
\item Отслеживать события попадания исполнения на контрольные точки
\item Отслеживать системные вызовы и другие события в наблюдаемом приложении
\end{itemize}

\end{frame}

\begin{frame}[fragile]{Пример уязвимого приложения}

\begin{small}
В мае 2011-го года была обнаружена уязвимость exim4.

Реализация тестировалась на аналогичной уязвимости в модельном приложении:
\end{small}

\scalebox{.50}{\lstinputlisting[label=samplecode2,caption=Модельный пример уязвимого приложения]{vuln.c}}

\end{frame}
\begin{frame}[fragile]{Пример разбиения уязвимого приложения}
\begin{figure}
\centering
\scalebox{.70}{\begin{tikzpicture}[->,<->,>=latex,grow=right,
    bd/.style={draw=blue!50!white, line width=1pt, dash pattern=on 1pt off 4pt on 6pt off 4pt,
                                rectangle, rounded corners}
    ]
\tikzstyle{state} = [draw,rounded corners, fill=white, rectangle, minimum height=3em, minimum width=5em, node distance=7em, font={\sffamily\small}]
\tikzstyle{stateEdgePortion} = [black,thick];
\tikzstyle{stateEdge} = [stateEdgePortion,->, minimum width=10em,node distance=8em];
\tikzstyle{shitEdge} = [stateEdgePortion,<->, minimum width=10em,node distance=8em];
\tikzstyle{edgeLabel} = [pos=0.5, text centered, font={\sffamily\small}];

\node[state, name=r] {read\_config};
\node[state, name=p, right of=r, node distance=4cm] {process\_input};

\draw[->] (r) edge[stateEdge, bend right=22.5] (p);
\draw[->] (p) edge[stateEdge, bend right=22.5, ] (p);

\node (111) [bd, inner sep=4mm, fit = (p)] {};
\node (222) [bd, inner sep=4mm, fit = (r)] {};

\node at(111) [right, inner sep=20mm] {test1\_init\_t};
\node at(222) [left, inner sep=3cm] {test1\_worker\_t};

\end{tikzpicture}

}
%  \caption{Разбиение тестового приложения.}
\end{figure}
\end{frame}

\begin{frame}{Результаты}
\begin{block}{Результаты}
\begin{itemize}

\item Разработан и реализован набор инструментов для разметки приложений
        контрольными точками.

\item Разработана и реализована подсистема ядра Линукс, позволяющая
        динамически переключать набор привилегий приложения при
        смене состояния.

\item Проведено экспериментальное исследование реализованного расширения SELinux
        на модельном примере, аналогичном уязвимой версии smtp-сервера
        exim4 2011 года.

\end{itemize}
\end{block}
\end{frame}

\begin{comment}
\begin{frame}{Характеристика реализации}
 
\begin{tabular}{| l | c | } 
\hline
  Язык программирования & C, Python\\ \hline
  Операционная система &  Linux \\ \hline  
  Кол-во строк кода & n \\ 
\hline
\end{tabular}
\end{frame}

\begin{frame}{Результаты}

\begin{block}{Результаты}
\begin{itemize}
 \item Проведен обзор и сравнительный анализ средств построения моделей нормального поведения приложений.
 \item Разработан метод автоматизированного построения нормального поведения приложений для ОС Linux и языка Си.
 \item Реализовано средство для приложений, написанных для ОС Linux на языке Си, позволяющие разбивать программу на набор блоков.
 \item Проведено испытание реализованного средства на уязвимом приложений.
\end{itemize}
\end{block}
\end{frame}
\end{comment}

\begin{frame}{Публикации по теме работы}

\begin{tiny}
\begin{block}{Публикации}
\begin{itemize}
\item Беззубцев С.О., Гамаюнов Д. Ю., Горнак Т.А., Сапожников А.В., Сахаров Ф.В., Контроль безопасного выполнения приложений с помощью поведенческих моделей. // Труды третьей всероссийской конференции "Методы и средства обработки информации", МГУ им. М.В.Ломоносова, 6-8 октября 2009, Москва, с. 439-444
\item Гамаюнов Д. Ю., Горнак Т.А., Сапожников А.В., Сахаров Ф.В, Торощин Э. С. Гранулярный контроль безопасности поведения приложений со стороны ядра Linux. // Информационно-методический журнал «Защита информации. Инсайд», №4, Санкт-Петербург, 2010, с. 54-58.


\end{itemize}
\end{block}

\begin{block}{Доклады на конференциях}
\begin{itemize}
\item Методы и средства обработки информации 2009, МСО-2009, Москва, 2009
\item Рускрипто-2010, Звенигородский р-н, 2010
\end{itemize}
\end{block}

\begin{block}{Исходный код проекта}
\begin{itemize}

\item \href{https://github.com/montekki/apathy}
    {https://github.com/montekki/apathy~--- реализованный модуль}
\item \href{https://github.com/montekki/linux-2.6-acedia/tree/acedia}
            {https://github.com/montekki/linux-2.6-acedia~--- модифицированное ядро Линукс}
\end{itemize}
\end{block}

\end{tiny}
\end{frame}

\begin{comment}
\end{comment}
% ----------------------------------------------------

\begin{comment}
  \begin{frame}{Спасибо за внимание. Вопросы?}
\begin{center}
\LARGE{Спасибо за внимание. Вопросы?}
\end{center}

   
  \end{frame}
\end{comment}

% ----------------------------------------------------
 

\end{document}

